\documentclass[11pt,wide]{mwart}

\usepackage[utf8]{inputenc}
\usepackage{polski}
\usepackage{graphicx}
\usepackage{hyperref}
\usepackage{amsmath,amssymb,amsfonts,amsthm,mathtools}
\usepackage{float}
\usepackage{enumerate}

\date{Wrocław, \today}
\title{\LARGE\textbf{Algorytmy i struktury danych}\\Zadanie 6 lista 1}
\author{Mikołaj Korobczak}

\begin{document}
\maketitle
\begin{center}
Prowadzący: mgr Adam Kunysz
\end{center}
\thispagestyle{empty}

\section{Treść zadania}
Dany jest niemalejący ciąg $n$ liczb dodatnich $a_1 \leq a_2 \leq \dots \leq a_n$. Wolno nam modyikować ten ciąg za pomocą następujących operacji: wybieramy dwa elementy $a_i, a_j$ spełniające $2a_i \leq a_j$ i wykreślamy je oba z ciągu. Ułóż algorytm obliczający, ile co najwyżej elementów możemy w ten sposób usunąć.

\section{Algorytm}
\noindent Algorytm będzie zachłanny: \\
Dane: $A[1,\dots,n]$
\begin{verbatim}
i = 1
j = (n div 2) + 1
counter = 0
while j <= n do
    if 2*A[i] <= A[j] then
        A[i] = -1
        A[j] = -1 
        counter = counter + 1
        i = i + 1
    endif
    j = j + 1
done
return counter
\end{verbatim}

\section{Dowód poprawności}
Weźmy dowolne rozwiązanie optymalne O. Załóżmy, że w tym rozwiązaniu usunięte zostało $k$ par. 
\newtheorem{theorem}{Lemat}
\begin{theorem}
\end{theorem}
Weźmy dwa zbiory $L$ i $R$ takie, że dla dowolnej usuwanej pary $(a_i, a_j)$ takiej, że $2a_i \leq a_j$, $a_i \in L$ i $a_j \in R$. Wtedy można zmienić te punkty w taki sposób, że:
\begin{enumerate}
\item $\forall_{a_i \in L} a_i < \min R$,
\item zbiór $L$ jest postaci $$ L = \{a_1, a_2, \dots, a_k\} $$
taki, że $\forall_{a_1 \leq a_i \leq a_k} a_i \in L$,
\item zbiór $R$ to
$$ R = \{a_{m + i}, a_{m+j}, \dots, a_{m+l}\} \qquad \text{dla } m=\lfloor \frac{n}{2} \rfloor + 1 \text{ i pewnych } 0 \leq i < j < l \text{.}$$ 
\end{enumerate}
 

\begin{proof} $ $\\
\begin{enumerate}
\item Weźmy dowolne dwie pary punktów $(a_i, a_i'), (a_j, a_j')$ takie, że $a_i' \leq a_j$. Wtedy $(a_i, a_j)$ oraz $(a_i', a_j')$ są poprawnymi parami więc można je poukładać w naszych zbiorach w ten sposób.
\item Weźmy dowolny punkt $a_i$ dla $i < k$ taki, że $a_i \notin L$. Weźmy najmniejszy punkt $a_j \in L$ taki, że $i < j \leq k$. Wtedy można punkt sparowany z nim sparować z naszym $a_i$, natomiast $a_j$ sparować z parą kolejnego punktu z $L$ i tak aż każdy będzie miał parę. Więc $a_i$ będzie należał do $L$.
\item Załóżmy, że zbiór $R$ składa się z punktów $(a_{i}, a_{j}, \dots, a_{k})$ takich, że $i < m$. Wtedy możemy zamienić te punkty na większe takie, że punkty zaczynają się od $a_{m+l}$ dla $l \geq 0$. Wiemy, że jest to możliwe ponieważ $k \leq \frac{n}{2}$. W ten sposób dostajemy żądany zbiór $R$. 
\end{enumerate}

\end{proof}

Korzystając z lematu 1 wystarczy pokazać, że nasz algorytm znajduje punktom $a_1, a_2, \dots, a_k$ parę.
Dowód będzie przez indukcję.
\begin{proof} $ $\\
Baza:\\
$a_1$ zostanie sparowane z najmniejszym $a_j$ takim, że $j \geq m = \lfloor \frac{n}{2} \rfloor + 1$, czyli znajdzie parę z najmniejszym punktem z $R$.\\
Krok:\\
Weźmy dowolne $i \leq k$ i załóżmy, że punkty $a_1, a_2, \dots, a_{i-1}$ znajazły pary, z kolejnymi najmniejszymi punktami $a_1', a_2', \dots, a_{i-1}'$ takimi, że są $\geq a_m$. Wtedy punkt $a_i$ znajdzie parę z najmniejszym punktem po punkcie $a_{i-1}'$, który spełni wymaganą nierówność. Taki punkt na pewno istnieje ponieważ w szczególności algorytm może brać punkty ze zbioru $R$, a wtedy napewno sparuje wszystkie punkty.
\end{proof}
Jednocześnie warto zaznaczyć, że algorytm nie może znaleść więcej niż $k$ par, ponieważ nie jest to możliwe sprawdzając za każdym razem zadaną nierówność.


\section{Złożoność czasowa i pamięciowa}
Złożoność pamięciowa to $O(1)$ ponieważ rezerwujemy pamięć tylko dla zmiennych $i, j, counter$. Jeżeli założymy, że musimy zarezerwować pamięć dla naszej tablicy (ja zakładam, że dostajemy tablicę już gdzieś w pamięci), to złożoność pamięciowa to $O(n)$.\\
Złożoność czasowa to $O(n)$ poniważ w najgorszym przypadku przebiegamy po naszej tablicy $A$ od $A[1]$ do $A[\lfloor \frac{n}{2} \rfloor ]$ oraz od $A[\lfloor \frac{n}{2} \rfloor + 1]$ do $A[n]$.

\end{document}
