\documentclass[11pt,wide]{mwart}

\usepackage[utf8]{inputenc}
\usepackage{polski}
\usepackage{graphicx}
\usepackage{hyperref}
\usepackage{amsmath,amssymb,amsfonts,amsthm,mathtools}
\usepackage{float}
\usepackage{enumerate}
\usepackage{ marvosym }

\date{Wrocław, \today}
\title{\LARGE\textbf{Algorytmy i struktury danych}\\Zadanie 6 lista 2}
\author{Mikołaj Korobczak}

\begin{document}
\maketitle
\begin{center}
Prowadzący: mgr Adam Kunysz
\end{center}
\thispagestyle{empty}

\section{Treść zadania}
Ułóż algorytm, który dla danego spójnego grafu $G$ oraz krawędzi $e$ sprawdza w czasie $O(n+m)$, czy krawędź $e$ należy do jakiegoś minimalnego drzewa spinającego grafu $G$. Możesz założyć, że wszystkie wagi krawędzi są różne.

\section{Własności MST}
Aby pokazać szukany algorytm i zrozumieć, że działa on poprawnie należy powiedzieć o kilku własnościach MST.
\begin{itemize}
\item Jeżeli założymy, że nasz graf $G$ ma krawędzie o parami różnych wagach to istnieje tylko jedno MST.
\item Jeżeli $e$ jest mostem w grafie $G$ to $e$ należy do MST.
\end{itemize}

\noindent Ponadto warto udowodnić pewien przydatny lemat.
\newtheorem{theorem}{Lemat}
\begin{theorem}
Dla dowolnej krawędzi $e$ w grafie $G$, $e$ jest w MST grafu $G$ wtedy i tylko wtedy, gdy dla dowolnego cyklu na którym leży krawędź $e$, krawędź ta nie jest krawędzią o największej wadze spośród wszystkim na tym cyklu.
\end{theorem}
\begin{proof}
$ $ \\
\begin{itemize}
\item[$\Rightarrow$] Załóżmy, że $e$ należy do MST. Weźmy cykl $C$ na którym leży $e$ i załóżmy nie wprost, że $e$ jest na nim krawędzią o największej wadze. Wiemy, że istnieje krawędź $f$ na cyklu $C$, której nie ma w MST. Jeżeli usuniemy z MST krawędź $e$ i dodamy krawędź $f$ to drzewo MST nie przstanie być spójne, ponieważ nadal będziemy patrzeć na cykl bez jednej krawędzi, a taka struktóra jest grafem spójnym. Nasze nowe drzewo MST ma mniejszą wagę niż poprzednio więc sprzeczność. \Lightning
\item[$\Leftarrow$] Załóżmy, że $e$ leży na cyklu $C$ i nie jest na nim krawędzią o największej wadze (i jeżeli leży na innych cyklach, to na nich też nie jest krawędzią o największej wadze). Rozważmy algorytm Kruskala (zakładam, że jest on poprawny). Algorytm ten będzie dodawał kolejne krawędzi z cyklu $C$ to MST dopóki cały cykl będzie spójny, ale nie będzie cyklem. Ponieważ taka sytuacja zajdzie dopiero, gdy zostanie jedna krawędź z tego cyklu poza MST i ponieważ algorytm ten dodaje krawędzie rosnąco po wagach, to krawędź $e$ zostanie dodana. Więc $e \in$ MST.
\end{itemize}
\end{proof}

\section{Algorytm}
Algorytm dostaje graf $G$ oraz krawędź $e$, jak również tablice wag $w[]$. \\
Nasz algorytm będzie zmodyfikowanym algorytmem DFS.
\begin{verbatim}
we := w[e]
REMOVE e FROM G
\end{verbatim}
Niech $e = \{v, u\}$. W tym momencie będziemy chodzić po grafie zgodnie z algorytmem DFS zaczynając od wierzchołka $v$ z poniższą modyfikacją. Niech $a$ to rozważany wierzchołek.
\begin{verbatim}
    visited[a] := 1
    f := //krawędź, która łączy a oraz następny wierzchołek do rozstrzygnięcia//
    weight := w[f]
    IF we < weight THEN
        //zignoruj tę krawędź//
    ENDIF
ENDWHILE
IF visited[u] == 1
    RETURN false
ELSE
    RETURN true
ENDIF
\end{verbatim}

Algorytm jak łatwo się domyślić zwraca \textit{true} jeżeli $e$ należy do MST i \textit{false} wpp.

\section{Dowód poprawności}
\begin{proof}

Rozważmy dwa przypadki:
\begin{itemize}
\item[•] $e$ nie jest w MST.\\ Wtedy leży na jakimś cyklu, na którym ma ma największą wagę. Wtedy nasz algorytm przejdzie po tym cyklu, bo wszystkie krawędzie mają mniejszą wagę od wagi $e$ i dojdziemy do drugiego wierzchołka $e$ i algorytm zwróci \textit{false}.
\item[•] $e$ jest w MST.\\ To oznacza, że albo $e$ jest mostem, albo dla każdego cyklu, na którym leży, jest krawędź o większej wade. Jeżeli jest mostem to po usunięciu tej krawędzi nie da się dojść do z jego wierzchołka tej krawędzi do drugiego więc algorytm zwróci \textit{true}. Jeżeli zaś zachodzi drugi przypadek to algorytm chcąc przejść dowolnym z tych cyklów do drugiego końca krawędzi $e$ natknie się na krawędź, która ma większą wagę od $e$ i nie będzie mógł dojść, więc algorytm zwróci \textit{true}.
\end{itemize}
\end{proof}

\end{document}