\documentclass[11pt, wide]{mwart}

\usepackage[utf8]{inputenc}
\usepackage{polski}
\usepackage{graphicx}
\usepackage{amsmath}
\usepackage{enumerate}

\begin{document}

Współrzędne planety na orbicie eliptycznej w czasie $t$ można obliczyć wzorem
\begin{equation*}
(a(\cos x - e), a \sqrt[]{1-e^2}\sin x),
\end{equation*}
gdzie $a$ jest półosią wielką elipsy, natomiast $e$ to mimośród orbity. Kąt $x$, zwany anomalią mimośrodową, \\
możemy obliczyć z równania Keplera,

\begin{equation*}
x - e \sin x = M \qquad (0 < |e| < 1),
\end{equation*}
gdzie $M$ to anomalia średnia, która jest dana wzorem $ M = 2 \pi t/T,$ przy czym $T$ oznacza okres orbitalny. \\

\begin{enumerate}[a)]
\item Pokaż, że dla każdych $e$, $M$ rozwiązanie $x = \alpha$ spełnia 
$M - |e| \leq \alpha \leq M + |e|$. Czy można poprawić to oszacowanie?
\item Rozwiąż równanie Keplera metodą bisekcji wykorzystując oszacowanie z poprzedniego podpunktu.
\item  Zaprogramuj prostą metodę iteracyjną,
\begin{equation*}
x_{n+1} = e \sin x_n + M, \quad x_0 = 0.
\end{equation*}
\item Zastosuj metodę Newtona do równania Keplera. Jak wybrać przybliżenie startowe?

\end{enumerate}
Wykonaj testy i porównaj zbieżność metod (b) oraz (c). Dane dotyczące planet Układu Słonecznego znajdziesz w Internecie. Jak dobrać przybliżenie startowe?
\\
\\
\textbf{Literatura:}
\begin{enumerate}[I.]
\item R. Esmaelzadeh, H. Ghadiri, Appropriate Starter for Solving the Kepler’s Equation, International Journal
of Computer Applications 89 (2014), 31–38.
\item G. R. Smith, A simple, efficient starting value for the iterative solution of Kepler’s equation, Celestial
Mechanics 19 (1979), 163–166.

\end{enumerate}

\end{document}